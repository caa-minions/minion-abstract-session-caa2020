\documentclass[a4paper]{article}
%\usepackage{simplemargins}

\usepackage[
	pdftitle={Our little minions, part 3: small tools with major impact},
	pdfsubject={Our little minions, part 3: small tools with major impact},
	pdfauthor={Ronald Visser, Moritz Mennenga, Florian Thiery},
	pdfkeywords={Minions, CAA, CAAOxford}
]{hyperref}

\input{doiCmd}
%\RequirePackage{doi}
%\usepackage[square]{natbib}
\usepackage{amsmath}
\usepackage{amsfonts}
\usepackage{amssymb}
\usepackage{graphicx}

\begin{document}
\pagenumbering{gobble}

\Large
 \begin{center}
Our little minions, part 3:\\ small tools with major impact\\ 

\hspace{10pt}

% Author names and affiliations
\large
Ronald Visser$^1$, Moritz Mennenga$^2$, Florian Thiery$^3$\\

\hspace{10pt}

\small  
$^1$ Saxion University of Applied Sciences, $^2$ Lower Saxony Institute of Historical Coastal Research, $^3$ R{\"o}misch-Germanisches Zentralmuseum\\
$^1$ r.m.visser@saxion.nl, $^2$ mennenga@nihk.de, $^3$ thiery@rgzm.de\\

\end{center}

\normalsize

In our daily work, small self-made scripts, home-grown small applications and small hardware devices significantly help us to get work done. These little helpers - 'little minions” - often reduce our workload or optimise our workflows, although they are not often presented to the outside world and the research community. Instead, we generally focus on presenting the results of our research and silently use our small tools during our research, without not even point to them, especially to the source code or building instructions.

This session will focus on these small helpers  - 'little minions” - and we invite researchers to share their tools, so that the scientific community may benefit and - perhaps - create spontaneously special minion interest groups.

As we have seen in the last years 'minion talks” there is a wide range of tools to be shared. These may be perfect examples for your own minion creation. A constantly expanding list of little minions can be found at https://github.com/caa-minions/minions.

At CAA international 2018 in Tübingen, a normal session (see S6 \cite{boa_caa_2018}) become spontaneously a 'Stand-up-Minion” lightning talk with a lot of nice pieces of source code, small tools and open/free software extensions for proprietary products. In 2018 we saw a tool for photogrammetric rectification of profile images of archaeological excavations, digital tools behind Bonify and database solutions for excavations.

In Krakow at CAA international 2019, a lot of little minions of various research domains were published to the research community (see O29 \cite{boa_caa_2019}). Martina Trognitz gave a deeper insight into Wikidata as a LOD minion addressing a 'Linked and Open Bibliography for Aegean Glyptic in the Bronze Age”. In terms of text mining, Ronald Visser showed his 'little text mining minion”. Florian Thiery and Allard Mees presented two small time minions to tame relative chronology and vague information in graph modelling using 'Taming Time Tools: Alligator and Academic Meta Tool”. A minion to do 'serial, fast and low cost 3D pottery on site documentation” was presented by Fanet Göttlich. Furthermore, Bart Vissers presented the minion 'CpyPst3D: a tool for direct exchange of 3D features with attributes between GIS, 3D-modeling environment and CAD”. Spontaneous minions were additions to profileAAR by Moritz Mennenga, the use of Heurist for collecting minions by Ian Johnson and a little minion by Gary Nobles to create a 3D volume object from point clouds of laser scans of excavation trenches.

This session aims at short presentations, lightning talks - aka 'minion talks” (max. 10 minutes including very short discussion) - of small coding pieces, software or hardware solutions, not only focusing on field work or excavation technology, associated evaluation or methodical approaches in data driven archaeology. Each 'minion talk” should explain the innovative character and mode of operation of the digital tool. The only restriction is that the software, source code and/or building instructions are open and are or will be freely available (e.g. GitHub, GitLab, etc.). Proprietary products cannot be presented, but only open and freely available tools designed for them.

We invite speakers to submit a short abstract including an introduction into the tool, the link to the repository to get access to the source code and an explanation which group of researchers could benefit from the little minion and how. The tools may address the following issues, but are not limited to, data processing tools and algorithms, measuring tools, digital documentation tools, GIS-Plugins, hands-on digital inventions (for excavations) and data driven tools (e.g. Linked Data, CSV, Big Data). After last years (pt.1 at CAA 2017 T{\"u}bingen and pt.2 at CAA 2018 Krakow) spontaneous success of 'Stand-up-Science”, you will also have the opportunity to spontaneously participate and demonstrate what you have on your stick or laptop. If you want to participate without an abstract in the spontaneous section of the session, please send an email to us (even shortly before the conference). Please come and spontaneously introduce your little minion!

The minion session is designed for technically interested researchers of all domains who want to present their small minions with the focus on the technical domain and also for researchers who want to get ideas which kind of little minions are available to help in their own research questions with the possibility to create spontaneously little minion special interest groups. All of us uses minions in the daily work and often they were build up multiple times. The reason is often that the focus in talks are on the projects and not on the technical details. We give these tools a slot that are considered too unimportant to be presented in the normal talks, but take important and extensive steps.

As an outcome of the session, all presented tools and links to code repositories will be available for the CAA research community. We will also collect all little minions in a 'CAA little minion catalogue” (http://littleminions.link) available for the public and extended in the future on a GitHub repository at https://github.com/caa-minions/minions.

\bibliographystyle{IEEEtran}
\bibliography{bib}

\end{document}